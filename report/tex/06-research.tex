
\chapter{Исследовательская часть}

\section{Критерии оценки методов сжатия изображений}

Для оценки методов сжатия изображений использовались следующие критерии.
\begin{enumerate}
    \item \textbf{Степень сжатия}: показывает, на сколько процентов от изначального размера файла удалось сжать изображение. Чем выше коэффициент, тем лучше удалось выполнить сжатие. При этом учитывается не только размер сжатого изображения, но и объем метаданных, необходимых для его восстановления. Степень сжатия рассчитывается по формуле.
    \begin{equation}
        \text{Степень сжатия} = \left(1 - \frac{\text{Размер сжатого изображения}}{\text{Размер исходного изображения}}\right) \times 100\%.
    \end{equation}
    \item \textbf{Размер информации для распаковки}: показывает, какую часть сжатого изображения занимает информация, необходимая для восстановления исходного файла. Чем выше этот показатель, тем большую долю от сжатого файла занимают метаданные. Большой объем информации для распаковки может не дать достичь высокой степени сжатия изображения.
\end{enumerate}

Для проведения исследования по выделенным критериям были выбраны изображения в формате BMP \cite{article-bmp}. Выбор файлов данного формата обусловлен следующими причинами:
\begin{itemize}
    \item Файлы в формате BMP хранят информацию о каждом пикселе изображения в исходном качестве без сжатия.
    \item BMP-файлы широко используются на практике в различных приложениях и системах.
    \item Формат BMP подходит для работы как с черно-белыми, так и с цветными изображениями.
\end{itemize}

\section{Сравнение разработанного метода сжатия с \mbox{аналогами}}

\subsection{Сравнение по степени сжатия изображений}

Результаты сравнения методов сжатия статических изображений без потерь по степени сжатия приведены в таблице \ref{tbl:compression_ratio} и продемонстрированы на рисунке \ref{img:comparison1}.
\captionsetup{justification=raggedleft,singlelinecheck=false}
\begin{table}[H]
    \centering
        \caption{Результаты сравнения методов сжатия изображений по степени сжатия}
        \label{tbl:compression_ratio}
        \begin{tabular}{|c|c|c|c|}
            \hline
            \textbf{Изображение}
            & \begin{minipage}[t]{4cm}\centering\textbf{Метод LZW, \%}\end{minipage} 
            & \begin{minipage}[t]{4cm}\centering\textbf{Разработанный метод, \%}\end{minipage}
            & \begin{minipage}[t]{4cm}\centering\textbf{Метод Хаффмана, \%}\end{minipage} \\ \hline
            sunrise.bmp &  83.01  &  75.29  &  69.91  \\ \hline
            mars.bmp    &  84.61  &  78.30  &  71.59  \\ \hline
            wheat.bmp   &  77.67  &  70.52  &  68.02  \\ \hline
            forest.bmp  &  44.52  &  54.23  &  67.77  \\ \hline
            girl.bmp    &  40.84  &  48.05  &  59.23  \\ \hline
        \end{tabular}
\end{table}

\imgs{comparison1}{h!}{0.66}{Сравнения методов сжатия статических изображений без потерь по степени сжатия}

На примерах видно, что степень сжатия зависит от типа данных: метод LZW лучше работает с длинными последовательностями одинаковых пикселей, показывая наивысшую степень сжатия для изображений \textit{sunrise.bmp} (83.01\%), \textit{mars.bmp} (84.61\%) и \textit{wheat.bmp} (77.67\%). Метод Хаффмана, напротив, демонстрирует лучшие результаты для изображений с неравномерным распределением цветов, таких как \textit{forest.bmp} (67.77\%) и \textit{girl.bmp} (59.23\%).

Гибридный метод является универсальным решением, которое позволяет минимизировать зависимость от особенностей входных изображений. Например, для изображения \textit{forest.bmp} он показывает степень сжатия 54.23\%, что выше, чем у метода LZW (44.52\%), но ниже, чем у метода Хаффмана (67.77\%). В то же время для изображения \textit{mars.bmp} его результат (78.30\%) уступает методу LZW (84.61\%), но превосходит метод Хаффмана (71.59\%).

Таким образом, гибридный метод обеспечивает более стабильный результат сжатия, выступая как компромиссное решение между высокой степенью сжатия и универсальностью.

\subsection{Сравнение по размеру информации для \mbox{распаковки} изображений}

Результаты сравнения методов сжатия статических изображений без потерь по количеству информации, необходимой для распаковки изображений, приведены в таблице \ref{tbl:info_to_decompress}.
\captionsetup{justification=raggedleft,singlelinecheck=false}
\begin{table}[H]
    \centering
        \caption{Результаты сравнения методов сжатия по размеру информации для распаковки изображений}
        \label{tbl:info_to_decompress}
        \begin{tabular}{|c|c|c|c|}
            \hline
            \textbf{Изображение}
            & \begin{minipage}[t]{4cm}\centering\textbf{Метод LZW, \%}\end{minipage} 
            & \begin{minipage}[t]{4cm}\centering\textbf{Разработанный метод, \%}\end{minipage}
            & \begin{minipage}[t]{4cm}\centering\textbf{Метод Хаффмана, \%}\end{minipage} \\ \hline
            sunrise.bmp &  0.53  &  41.20  &  0.51  \\ \hline
            mars.bmp    &  0.77  &  41.81  &  0.72  \\ \hline
            wheat.bmp   &  0.76  &  39.07  &  0.88  \\ \hline
            forest.bmp  &  0.25  &  33.50  &  0.73  \\ \hline
            girl.bmp    &  2.72  &  39.05  &  6.59  \\ \hline
        \end{tabular}
\end{table}

\clearpage

Визуализация результатов сравнения приведена на рисунке \ref{img:comparison2}.

\imgs{comparison2}{h!}{0.66}{Сравнение методов сжатия статических изображений без потерь по количеству информации для распаковки}

Гибридный метод сжатия, сочетающий в себе алгоритмы LZW и Хаффмана, требует больше информации для распаковки изображения по сравнению с отдельными методами. Это связано с необходимостью сохранения данных, используемых на обоих этапах сжатия. Например, после применения LZW сохраняются уникальные пиксели изображения и размер кода LZW, а на этапе Хаффмана добавляются таблица частот символов и количество байт, необходимых для сохранения частоты одного символа. Каждый из этих компонентов увеличивает объем метаданных. Как видно из таблицы \ref{tbl:info_to_decompress}, для гибридного метода доля информации для распаковки составляет от 33.50\% до 41.81\%, тогда как для LZW и Хаффмана этот показатель не превышает 6.59\%.

\section*{Выводы}

Несмотря на больший объем метаданных в сжатом файле, гибридный метод обеспечивает более стабильный результат сжатия за счет сильных сторон обоих алгоритмов. LZW эффективно сжимает повторяющиеся последовательности (например, однородные участки изображений), а Хаффман оптимизирует кодирование частых символов. Это позволяет гибридному методу адаптироваться к разным типам изображений, минимизируя зависимость от их структуры. Например, для изображения forest.bmp гибридный метод показывает степень сжатия 54.23\%, что лучше, чем у LZW (44.52\%), и близко к результату Хаффмана (67.77\%). Для изображения mars.bmp метод LZW демонстрирует степень сжатия 84.61\%, что значительно лучше, чем у метода Хаффмана (71.59\%). Гибридный метод показывает промежуточный результат (78.30\%), сохраняя универсальность. Таким образом, компромисс между объемом метаданных и универсальностью делает гибридный метод подходящим для задач, где важна стабильность, а не абсолютная минимизация размера файла.
