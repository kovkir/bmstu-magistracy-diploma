\chapter{Конструкторская часть}

\section{Требования к разрабатываемому методу сжатия изображений}

Для гибридного метода сжатия изображений были выдвинуты следующие требования:
\begin{itemize}
    \item на вход разрабатываемый метод должен получать путь до файла со статическим изображением и путь до директории, куда необходимо сохранить сжатый файл;
    \item результатом работы метода должно стать сжатое изображение, сохраненное в указанной директории;
    \item сжатое изображение должно содержать всю информацию, необходимую для его восстановления;
    \item сжатие должно сохранять всю информацию об исходном изображении.
\end{itemize}

\section{Проектирование метода сжатия изображений}

Разрабатываемый метод сжатия статических изображений будет представлять собой гибридный метод на основе алгоритма сжатия Хаффмана. Улучшение данного метода будет производиться за счет первичной обработки изображения другим алгоритмом сжатия \cite{my-article}. В качестве такого метода был выбран LZW из-за:
\begin{itemize}
    \item возможности кодирования данных за один проход;
    \item отсутствия необходимости в таблице частот пикселей сжимаемого изображения.
\end{itemize}

Метод LZW удаляет избыточность из последовательности пикселей изображения. Он заменяет повторяющиеся подстроки уникальными кодами, что значительно уменьшает размер изображения и приводит к уменьшению размера дерева кодов Хаффмана. 

Разрабатываемый гибридной метод сжатия будет состоять из следующих этапов:
\begin{itemize}
    \item получение данных сжимаемого изображения в виде байтовой строки, которая будет использоваться в качестве входных данных для алгоритма LZW;
    \item выполнение первичного сжатия изображения алгоритмом LZW;
    \item нахождение таблицы частот символов;
    \item построение дерева кодов Хаффмана на основе вычисленной таблицы частот символов;
    \item выполнение повторного сжатия изображения алгоритмом Хаффмана;
    \item создание файла с сжатым изображением и информацией для его распаковки.
\end{itemize}

Таким образом, использование первичной обработки изображения в гибридном методе сжатия позволяет подготовить данные для метода Хаффмана путем уменьшения количества обрабатываемых символов. Такой подход приводит к более эффективному сжатию Хаффмана и, следовательно, к более высокой общей степени сжатия.

Диаграмма уровня A1 (рисунок \ref{img:idef0_a1.pdf}) иллюстрирует общую структуру гибридного метода сжатия: преобразование изображения в байтовую строку, этап сжатия и создание итогового файла с сжатым изображением.
\imgs{idef0_a1.pdf}{h!}{0.65}{Детализированная IDEF0-диаграмма гибридного метода сжатия изображений первого уровня}

Диаграмма уровня A2 (рисунок \ref{img:idef0_a1.pdf}) раскрывает детали этапа сжатия, выделяя первичную обработку изображения методом LZW, построение таблицы частот символов, генерацию дерева Хаффмана и повторное кодирование методом Хаффмана.
\imgs{idef0_a2.pdf}{h!}{0.6}{Детализированная IDEF0-диаграмма гибридного метода сжатия изображений уровня A2}

\section{Требования к разрабатываемому ПО}

Для демонстрации работы гибридного метода необходимо разработать ПО со следующими требованиями:
\begin{itemize}
    \item взаимодействие пользователя с ПО должно осуществляться с помощью графического интерфейса;
    \item необходимо предусмотреть возможность выбора сжимаемого изображения через файловый менеджер;
    \item необходимо предусмотреть возможность восстановления сжатых изображений;
    \item пользователь должен иметь возможность сравнения гибридного метода сжатия изображений с базовыми, на основе которых он был разработан;
    \item сравнение должно проводиться по степени сжатия исходного файла, а также по размеру информации для распаковки в сжатом изображении.
\end{itemize}

Также необходимо подготовить список тестовых изображений для сжатия и положить их в директорию \texttt{input\_data}.


\section{Схемы разрабатываемого гибридного метода сжатия изображений}

\subsection{Схема гибридного метода сжатия}

Схема гибридного метода сжатия статических изображений представлена на рисунке \ref{img:hybrid.pdf}. Она состоит из шести основных пунктов, три из которых далее будут рассмотрены подробно.
\imgs{hybrid.pdf}{h!}{0.95}{Схема гибридного метода сжатия изображений}

\clearpage
\subsection{Схема метода LZW для первичного сжатия данных}

Схема метода первичного сжатия LZW представлена на рисунке \ref{img:lzw.pdf}. На данном этапе происходит удаление избыточности данных и уменьшение количества обрабатываемых символов.
\imgs{lzw.pdf}{h!}{0.92}{Схема метода LZW для первичного сжатия изображений}

\clearpage
\subsection{Схема построения дерева кодов Хаффмана}

Схема построения дерева кодов Хаффмана на основе таблицы частот символов представлена на рисунке \ref{img:huffman.pdf}. На основе этого дерева будет поизведено сжатие байтовой строки, полученной на этапе первичного сжатия изображения методом LZW.
\imgs{tree.pdf}{h!}{0.93}{Схема построения дерева кодов Хаффмана}

\clearpage
\subsection{Схема метода Хаффмана для повторного сжатия данных}

Схема метода Хаффмана для повторного сжатия данных представлена на рисунке \ref{img:huffman.pdf}. Это основной этап гибридного метода, в результате которого будет получена байтовая строка с итоговым сжатым изображением.
\imgs{huffman.pdf}{h!}{0.9}{Схема метода Хаффмана для повторного сжатия данных}

\section*{Вывод}

В данном разделе были предъявлены требования к разрабатываемому методу сжатия статических изображений и к разрабатываемому ПО, произведено проектирование метода сжатия. Для первичного сжатия, удаления избыточности и уменьшения количества обрабатываемых символов был выбран метод LZW. Кроме того, в данном разделе были построены схемы для реализации гибридного метода сжатия.
