\chapter{Технологическая часть}

\section{Используемые программные средства для реализации метода}

В качестве языка программирования был выбран \texttt{Python} \cite{Python}. Для \texttt{Python} существует большое количество библиотек и документация на русском языке, а сам язык поддерживает объектно-ориентированную парадигму программирования.

При создании графического интерфейса для программного обеспечения была использована библиотека \texttt{tkinter} \cite{Tkinter}. Она является кроссплатформенной и включена в стандартную библиотеку языка \texttt{Python} в виде отдельного модуля. Для визуализации сравнения работы методов сжатия изображений использовалась библиотека \texttt{matplotlib} \cite{Matplotlib} с следующими модулями:
\begin{itemize}
    \item \texttt{matplotlib.pyplot} \cite{Pyplot} --- модуль, предоставляющий функции для создания графиков и визуализации данных, использовался для построения столбчатых диаграмм при сравнении методов сжатия изображений;
    \item \texttt{matplotlib.offsetbox} \cite{OffsetBox} --- модуль, предоставляющий возможность размещения текстовых и графических элементов на построенных графиках, использовался для добавления сжимаемых изображений под диаграммами сравнения методов.
\end{itemize}

Для работы с массивами битов при сжатии данных методом Хаффмана была использована библиотека \texttt{bitarray} \cite{Bitarray}, для отображения прогресса этапов сжатия и распаковки изображения использовалась библиотека \texttt{progress} \cite{Progress}. Для получения списка файлов, доступных для сжатия, был использован модуль \texttt{subprocess} \cite{Subprocess}.


\section{Формат входных и выходных данных}

В качестве входных данных разработанный программный комплекс получает путь до изображения в формате BMP, TIFF, PNG или JPEG, а также путь до директории, куда необходимо сохранить сжатое и распакованное изображения. Также пользователю предоставляется возможность выбрать один из трех методов сжатия: LZW, Хаффман или гибридный метод, разработанный в данной работе.

На выходе в директории с результатами создаются два файла с расширениями .bin (для сжатого изображения) и .bmp (для распакованного). В консоль выводится подробная информация об этапах сжатия и распаковки изображения, размеры исходного и полученного файлов, а также итоговая степень сжатия.

\section{Структура разработанного ПО}

\subsection{Описание этапов гибридного метода сжатия}

Реализация гибридного метода сжатия статических изображений без потерь состоит из следующих основных этапов.
\begin{enumerate}
    \item Первичное сжатие изображения методом lZW.
    \item Создание таблицы частот символов.
    \item Построение дерева Хаффмана.
    \item Применение метода сжатия Хаффмана к байтовой строке, полученной после первого этапа алгоритма.
    \item Создание файла с сжатым изображением и информацией для его распаковки.
\end{enumerate}

На первом этапе метода проводится первичное сжатие изображения методом LZW. В процессе обработки пикселей входного изображения создается словарь повторяющихся цепочек байт. Выделенные последовательности пикселей заменяются на уникальные коды фиксированной длины. Размер таких кодов зависит от количества заменяемых последовательностей (чем длиннее код, чем больше цепочек байт можно заменить на первом этапе метода). При распаковки сжатого изображения используется тот же словарь повторяющихся цепочек пикселей.

На втором этапе подсчитывается количество каждого уникального кода в полученной байтовой строке, строится таблица частот символов.

Третий этап включает в себя построение дерева Хаффмана, задача которого заключается в присвоении часто встречающимся символам более коротких кодов, а редко встречающимся --- более длинных. В отличии от классического дерева Хаффмана, в разработанном методе за один символ принимается не один байт, а уникальный код, состоящий из заданного числа байт.

На четвертом этапе происходит применение метода сжатия Хаффмана к байтовой строке, полученной после первого этапа алгоритма. На основе построенного дерева каждой цепочке байт (уникальному коду из метода LZW) присваивается код Хаффмана переменной длины, который за счет уникального префикса может быть однозначно декодирован.

На заключительном этапе метода происходит формирование байтовой строки со сжатым изображением и информацией для его распаковки, которая включает в себя таблицу частот символов (для восстановления дерева Хаффмана) и уникальные пиксели исходного изображения (для воссоздания словаря повторяющихся цепочек байт). Полученная байтовая строка является результатом сжатия статического изображения разработанным гибридным методом.

\subsection{Описание модулей разработанного ПО}

UML-диаграмма \cite{UML} компонентов разработанного программного обеспечения представлена на рисунке \ref{img:uml}. Она показывает структуру зависимостей между основными модулями программы.

\imgs{uml}{h!}{1}{UML-диаграмма компонентов разработанного ПО}

Разработанное программное обеспечение состоит из следующих модулей.
\begin{enumerate}
    \item \textbf{Модуль сжатия и распаковки изображений}. Реализует основной функционал сжатия и распаковки изображений разработанным гибридным методом (листинг \ref{lst:compression}). Основной класс модуля, \texttt{Compression}, отвечает за выполнение всех этапов сжатия и восстановления данных. При сжатии входное изображение представляется в виде байтовой строки, к которой применяется выбранный метод (LZW, Хаффман или гибридный), после чего формируется файл с сжатым изображением и метаданными для его распаковки.
    \item \textbf{Модуль сжатия и распаковки методом LZW}. Реализует алгоритм сжатия и распаковки данных методом LZW (листинг \ref{lst:lzw}). Включает создание словаря повторяющихся последовательностей байт и их замену уникальными кодами фиксированной длины. При сжатии генерирует сжатую байтовую строку и список уникальных пикселей, а при распаковке восстанавливает исходные данные на основе словаря, воссозданного по списку пикселей.
    \item \textbf{Модуль сжатия и распаковки методом Хаффмана}. Реализует алгоритм сжатия и распаковки данных методом Хаффмана (листинг \ref{lst:huffman}). Включает построение таблицы частот символов, создание дерева Хаффмана и генерацию кодов переменной длины. При сжатии преобразует данные в битовую строку на основе построенного дерева, а при распаковке восстанавливает исходные данные на основе сохраненной таблице частот символов.
    \item \textbf{Модуль построения дерева Хаффмана}. Предоставляет вспомогательные классы и функции для работы с деревом Хаффмана (листинг \ref{lst:tree}). Включает создание узлов дерева, объединение их на основе частот символов и генерацию кодов Хаффмана. Модуль используется в \texttt{huffman.py} для построения дерева и кодирования данных, а также для восстановления исходной информации при распаковке.
    \item \textbf{Модуль взаимодействия с пользователем}. Реализует графический интерфейс (листинг \ref{lst:window}) с использованием библиотеки \texttt{tkinter}. Позволяет пользователю выбрать входное изображение, метод сжатия (LZW, Хаффман или гибридный), а также директорию для сохранения результатов. Модуль отображает прогресс выполнения операций, результаты сжатия и распаковки, а также предоставляет визуализацию сравнения методов сжатия.
\end{enumerate}

\section{Результаты работы ПО}

Разработанное программное обеспечение представляет собой приложение с графическим интерфейсом (рисунок \ref{img:interface}), предоставляющее возможность выбора исходного изображения, метода сжатия и директории для сохранения результатов. Пользователь может сжать и распаковать выбранное изображение, посмотреть результаты сравнения доступных методов сжатия, а также получить информацию о данной программе.
Подробная информация об этапах сжатия и распаковки выводится как в консоль, так и в окно программы.

\imgs{interface}{h!}{0.37}{Интерфейс программы для сжатия изображений}

\clearpage

Для демонстрации работы гибридного метода сжатия было выбрано изображение \texttt{sunrise.bmp}, представленное на рисунке \ref{img:input_sunrise}.

\imgs{input_sunrise}{h!}{0.68}{Исходное изображение восхода солнца}

После сжатия файла пользователю выводится график сравнения размеров полученного изображения с исходным в виде столбчатой диаграммы (рисунок \ref{img:sunrise_compression_result}). Данный график позволяет оценить коэффициент сжатия файла и размер методанных, необходимых для распаковки изображения.

\imgs{sunrise_compression_result}{h!}{0.66}{Результаты сравнения размеров сжатого изображения с исходным (изображение восхода солнца)}

Подробная информация об этапах сжатия и распаковки изображения восхода солнца продемонстрирована в листинге \ref{lst:sunrise_compression_results}.

\mylisting[text]{sunrise_compression_results.txt}{}{Результаты сжатия и распаковки входного изображения восхода солнца гибридным методом}{sunrise_compression_results}{}

Распакованное изображение \texttt{sunrise.bmp} с сохранением всех деталей исходного представлено на рисунке \ref{img:output_sunrise}.

\imgs{output_sunrise}{h!}{0.7}{Восстановленное изображение восхода солнца}

Для изображения \texttt{sunrise.bmp} исходный размер файла составил 720.1 КБ, а количество различных пикселей в изображении --- 212. После сжатия разработанным гибридным методом (LZW + Хаффман) размер сжатого изображения составил 104.6 КБ, а информация для его распаковки заняла 73.3 КБ, что в сумме дало размер сжатого файла 177.9 КБ. Степень сжатия файла составила 75.29\%. 

Среднее число пикселей в цепочках, созданных методом LZW, составило 3.93, а среднее количество повторных использований цепочек байт --- 2.51. Распаковка файла успешно восстановила исходное изображение с размером 720.1 КБ.

\clearpage


Для следующего примера работы гибридного метода сжатия было выбрано изображение \texttt{girl.bmp}, представленное на рисунке \ref{img:input_girl}.

\imgs{input_girl}{h!}{0.7}{Исходное изображение девушки}

Сравнение размеров сжатого файла с исходным для изображения \texttt{girl.bmp} представлено на рисунке \ref{img:girl_compression_result}. 

\imgs{girl_compression_result}{h!}{0.66}{Результаты сравнения размеров сжатого изображения с исходным (изображение девушки)}

Подробная информация об этапах сжатия и распаковки изображения девушки продемонстрирована в листинге \ref{lst:sunrise_compression_results}.

\mylisting[text]{girl_compression_results.txt}{}{Результаты сжатия и распаковки входного изображения девушки гибридным методом}{girl_compression_results}{}

Распакованное изображение \texttt{girl.bmp} с сохранением всех деталей исходного представлено на рисунке \ref{img:output_girl}.

\imgs{output_girl}{h!}{0.7}{Восстановленное изображение девушки}

Для изображения \texttt{girl.bmp} исходный размер файла составил 491.3 КБ, а количество уникальных пикселей в изображении --- 2643. После применения гибридного метода сжатия (LZW + Хаффман) размер сжатого изображения составил 155.5 КБ, а данные для его восстановления заняли 99.7 КБ, что в сумме дало общий размер сжатого файла 255.3 КБ. Степень сжатия составила 48.05\%.

Средняя длина цепочек пикселей, сформированных методом LZW, составила 1.69, а среднее количество повторений цепочек байт --- 4.11. Размер таблицы частот символов, использованной для построения дерева Хаффмана, составил 22.9 КБ. Максимальная частота символа в таблице составила 98, что потребовало 1 байт для хранения частоты. Распаковка файла успешно восстановила исходное качество изображения.

\section*{Вывод}

В данном разделе были рассмотрены используемые программные средства реализации метода, описан формат входных и выходных данных, описана реализация гибридного метода сжатия статических изображений и приведены результаты работы программы. Также было представлено описание структуры разработанного ПО.

В примере изображения с восходом солнца степень сжатия в 1.56 раз больше, чем в изображении с девушкой (75.29\% против 48.05\%). Это связано с тем, что в файле \texttt{sunrise.bmp} цепочки байт, полученные на этапе обработки изображения методом LZW, в среднем содержат больше пикселей на 133\% (3.93 против 1.69). Также в изображении \texttt{girl.bmp} больше уникальных пикселей (2643 против 212, то есть в 12.47 раз). Эти факторы способствуют более эффективному сжатию изображения с восходом солнца разработанным гибридным методом.
