\maketableofcontents

\intro

Методы сжатия статических изображений активно применяются для хранения и передачи растровых изображений. За счет уменьшения размера файла, методы сжатия позволяют достичь увеличения скорости передачи данных, а также уменьшения занимаемого на диске места.

Например, одной из областей применения методов сжатия изображений является медицина, где важно передавать снимки КТ, МРТ, УЗИ, рентгеновские снимки в медицинских информационных системах с привлечением минимальных ресурсов.

Также методы сжатия изображений активно применяются в интернет-магазинах, где скорость загрузки снимков товаров на странице с ассортиментом является ключевой.

Задача сжатия изображений остается актуальной, так как файлов с ними с каждым днем становится все больше. Совершенствование методов сжатия и разработка новых алгоритмов остается важной задачей для обеспечения эффективного хранения и передачи изображений.

Целью выпускной квалификационной работы является разработка метода сжатия статических изображений без потерь на основе алгоритма Хаффмана.

Для достижения поставленной цели необходимо выполнить следующие задачи:

\begin{itemize}
    \item провести аналитический обзор известных методов сжатия статических изображений;
    \item разработать метод сжатия статических изображений без потерь на основе алгоритма Хаффмана;
    \item разработать программное обеспечение для демонстрации работы созданного метода;
    \item провести сравнение разработанного метода с аналогами по степени сжатия изображений.
\end{itemize}
