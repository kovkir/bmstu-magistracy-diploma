\addcontentsline{toc}{chapter}{РЕФЕРАТ}
\chapter*{РЕФЕРАТ}

Рассчетно-пояснительная записка к выпускной квалификационной работе <<Метод сжатия статических изображений без потерь на основе алгоритма Хаффмана>> содержит \begin{NoHyper}\pageref{LastPage}\end{NoHyper} страниц, 4 части, \totfig~ рисунков, 5 таблиц и список используемых источников из 26 наименования. 

Ключевые слова: сжатие изображений, алгоритм Хаффмана, дерево Хаффмана алгоритм LZW, сжатие без потерь, статические изображения, bmp-файлы.

Объект разработки --- метод сжатия статических изображений без потерь.

Цель работы: разработать метод сжатия статических изображений без потерь на основе алгоритма Хаффмана.

В первой части работы рассмотрены основные методы сжатия данных без потерь. Сформулированы критерии сравнения методов сжатия. Выполнен сравнительный анализ исследуемых методов по выделенным критериям. Описана формальная постановка задачи в виде IDEF0-диаграммы.

Во второй части разработан метод сжатия статических изображений на основе алгоритма Хаффмана. Описаны основные особенности предлагаемого метода. Изложены ключевые этапы метода в виде схем алгоритмов.

В третьей части обоснован выбор программных средств для реализации предложенного метода. Описан формат входных и выходных данных. Разработано программное обеспечение, реализующее описанный метод. Описано взаимодействие пользователя с программным обеспечением.

В четвертой части в рамках исследования проведено сравнение разработанного метода сжатия статических изображений без потерь с рассмотренными аналогами. В качестве критериев сравнения использовались полученная степень сжатия файла и размер информации, необходимой для распаковки изображения.

Разработанный метод сжатия статических изображений без потерь может применяться в системах хранения и передачи данных, где важна высокая степень сжатия изображений без потери их качества.
