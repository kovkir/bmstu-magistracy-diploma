\conclusion

В ходе выполнения выпускной квалификационной работы разработан метод сжатия статических изображений без потерь на основе алгоритма Хаффмана. Поставленная цель была достигнута.

Проведен аналитический обзор известных методов сжатия статических изображений. Были рассмотрены алгоритмы сжатия без потерь (RLE, LZW, унарное кодирование, Хаффман и арифметическое кодирование) и с потерями (JPEG, Wavelet, фрактальное сжатие), проведено сравнение методов по выделенным критериям. Рассмотрены основные цветовые модели изображений (RGB, RGBA, CMYK, LAB, HSB).

Разработан метод сжатия статических изображений без потерь на основе алгоритма Хаффмана. Он представляет собой гибридную реализацию, где для первичной обработки изображения используется словарный метод LZW.

Разработано программное обеспечение для демонстрации работы созданного метода. Реализованная программа представляет собой графическое приложение, предоставляющее пользователю возможность выбора изображения для сжатия и распаковки. 

Программное обеспечение позволяет провести сравнение гибридного метода с Хаффманом и LZW по степени сжатия изображений и размеру информации, необходимой для их распаковки. Гибридный метод показал более стабильный результат сжатия, минимизировав зависимость от особенностей входных изображений, несмотря на большую величину метаданных, необходимых для восстановления исходных файлов.
