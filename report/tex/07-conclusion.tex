\conclusion

В ходе выполнения выпускной квалификационной работы надо было разработать метод сжатия статических изображений без потерь на основе алгоритма Хаффмана. Поставленная цель была достигнута. Также были выполнены следующие задачи.

\begin{enumerate}
  \item Проведен аналитический обзор известных методов сжатия статических изображений. Были рассмотрены алгоритмы сжатия без потерь (RLE, LZW, унарное кодирование, Хаффман и арифметическое кодирование) и с потерями (JPEG, Wavelet, фрактальное сжатие), проведено сравнение методов по выделенным критериям. Также были рассмотрены основные цветовые модели изображений (RGB, RGBA, CMYK, LAB, HSB).
  \item Разработан метод сжатия статических изображений без потерь на основе алгоритма Хаффмана. Он представляет собой гибридную реализацию, где для первичной обработки изображения используется словарный метод LZW.
  \item Разработано программное обеспечение для демонстрации работы созданного метода. Реализованная программа представляет собой графическое приложение, предоставляющее пользователю возможность выбора изображения для сжатия и распаковки. Также данное ПО позволяет провести сравнение гибридного метода с Хаффманом и LZW.
  \item Проведено сравнение разработанного метода с аналогами по степени сжатия изображений. Гибридный метод показал более стабильный результат сжатия, минимизировав зависимость от особенностей входных изображений.
\end{enumerate}
